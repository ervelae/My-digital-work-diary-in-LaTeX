% ---- File: preamble.tex ----
% Shared packages and global configurations for all templates.
% This file is \input by main.tex, Weekly_template.tex, Monthly_template.tex, and Annual_template.tex

% ---- Packages ----
\usepackage{hyperref}
\usepackage{pgffor}
\usepackage{pgfmath}
\usepackage{pgfcalendar}
\usepackage{tikz}
\usepackage{tabularx}
\usepackage{parskip}
\usepackage{xcolor}
\usepackage{amssymb}
\usepackage{ifthen}
\usepackage{etoolbox}
\usepackage{colortbl}
\usepackage{array}
\usepackage{libertine}  % or lmodern, libertine, garamond, etc.
\usepackage[paperwidth=288mm,paperheight=162mm,margin=2mm]{geometry}

% Geometry (can be overridden in main.tex if needed)
\providecommand{\PaperWidth}{288mm}
\providecommand{\PaperHeight}{162mm}
\providecommand{\PaperMargin}{2mm}

% TikZ libraries
\usetikzlibrary{calendar,shapes.multipart}

% ---- Hyperref setup ----
\hypersetup{colorlinks=true,linkcolor=blue!45!black}

% ---- Layout & formatting ----
\setlength{\parindent}{0pt}

% ---- Global Configurations ----
% Year for which you want automatic mapping:
\providecommand{\DiaryYear}{2029} % <-- set the year here

% Mode: "iso" (default) or "jan1"
% - "iso": ISO week numbering (week 1 is the ISO week)
%     mapping rule: week assigned to the month containing the week's Monday.
% - "jan1": week 1 is the week that contains Jan 1, weeks start on Monday; mapping by Monday's month.
\providecommand{\WeekMode}{iso}

% ---- Helper Commands ----

% Month names
\providecommand{\MonthName}[1]{% 1..12
  \ifcase#1\relax\or January\or February\or March\or April\or May\or June\or July\or August\or September\or October\or November\or December\fi}

% Week date calculation helpers
\newcount\julianfirstweek
\newcount\wdayfirstweek
\newcount\mondayfirstweek
\newcount\julianmonday
\newcount\juliansunday

% Calculate week start date (Monday) given ISO week number
% Sets \WeekStartDay, \WeekStartMonth, \WeekStartYear
\providecommand{\CalcWeekStartDate}[2]{%
  \pgfcalendardatetojulian{#1-01-04}{\julianfirstweek}%
  \pgfcalendarjuliantoweekday{\julianfirstweek}{\wdayfirstweek}%
  \mondayfirstweek=\julianfirstweek
  \advance\mondayfirstweek by -\wdayfirstweek%
  \pgfmathtruncatemacro{\weekoffset}{(#2-1)*7}%
  \julianmonday=\mondayfirstweek
  \advance\julianmonday by \weekoffset\relax%
  \pgfcalendarjuliantodate{\julianmonday}{\tmpyear}{\tmpmonth}{\tmpday}%
  \xdef\WeekStartDay{\tmpday}%
  \xdef\WeekStartMonth{\tmpmonth}%
  \xdef\WeekStartYear{\tmpyear}%
}

% Calculate week end date (Sunday) given ISO week number
% Sets \WeekEndDay, \WeekEndMonth, \WeekEndYear
\providecommand{\CalcWeekEndDate}[2]{%
  \pgfcalendardatetojulian{#1-01-04}{\julianfirstweek}%
  \pgfcalendarjuliantoweekday{\julianfirstweek}{\wdayfirstweek}%
  \mondayfirstweek=\julianfirstweek
  \advance\mondayfirstweek by -\wdayfirstweek%
  \pgfmathtruncatemacro{\weekoffset}{(#2-1)*7 + 6}%
  \juliansunday=\mondayfirstweek
  \advance\juliansunday by \weekoffset\relax%
  \pgfcalendarjuliantodate{\juliansunday}{\tmpyear}{\tmpmonth}{\tmpday}%
  \xdef\WeekEndDay{\tmpday}%
  \xdef\WeekEndMonth{\tmpmonth}%
  \xdef\WeekEndYear{\tmpyear}%
}

% Format week date range
% Sets \WeekDateRange
\providecommand{\FormatWeekDates}[2]{%
  \CalcWeekStartDate{#1}{#2}%
  \CalcWeekEndDate{#1}{#2}%
  \xdef\WeekDateRange{\WeekStartDay.\WeekStartMonth. - \WeekEndDay.\WeekEndMonth.}%
}


% Build week-to-month mapping
\foreach \w in {1,...,53}{%
  \CalcWeekStartDate{\DiaryYear}{\w}%
  \xdef\WeekMonthNum{\the\numexpr\WeekStartMonth\relax}%
  \ifcsdef{WeeksForMonth\WeekMonthNum}%
    {\csxdef{WeeksForMonth\WeekMonthNum}{\csuse{WeeksForMonth\WeekMonthNum},\w}}%
    {\csxdef{WeeksForMonth\WeekMonthNum}{\w}}%
}


% Build week-to-date mapping for quick lookup
\foreach \w in {1,...,53}{%
  \CalcWeekStartDate{\DiaryYear}{\w}%
  \csxdef{WeekStartDay\w}{\WeekStartDay}%
  \csxdef{WeekStartMonth\w}{\WeekStartMonth}%
}


\newcommand{\ClickBox}[2]{% #1=label target, #2=visible text
  \hyperref[#1]{\kern2pt\raisebox{0pt}[13pt][9pt]{\mbox{#2}}\kern2pt}%
}
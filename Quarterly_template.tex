% ---- File: Quarterly_template.tex ----
% A single-instance annual overview page (compiles standalone).
% When included from main.tex, define \INCLUDEMODE to suppress preamble.
\ifdefined\INCLUDEMODE\else
  \documentclass[11pt]{article}
    \usepackage{tabularx}
  \usepackage{calc}
  % ---- File: preamble.tex ----
% Shared packages and global configurations for all templates.
% This file is \input by main.tex, Weekly_template.tex, Monthly_template.tex, and Annual_template.tex

% ---- Packages ----
\usepackage{hyperref}
\usepackage{pgffor}
\usepackage{pgfmath}
\usepackage{pgfcalendar}
\usepackage{tikz}
\usepackage{tabularx}
\usepackage{parskip}
\usepackage{xcolor}
\usepackage{amssymb}
\usepackage{ifthen}
\usepackage{etoolbox}
\usepackage{colortbl}
\usepackage{array}
\usepackage{libertine}  % or lmodern, libertine, garamond, etc.
\usepackage[paperwidth=288mm,paperheight=162mm,margin=2mm]{geometry}

% Geometry (can be overridden in main.tex if needed)
\providecommand{\PaperWidth}{288mm}
\providecommand{\PaperHeight}{162mm}
\providecommand{\PaperMargin}{2mm}

% TikZ libraries
\usetikzlibrary{calendar,shapes.multipart}

% ---- Hyperref setup ----
\hypersetup{colorlinks=true,linkcolor=blue!45!black}

% ---- Layout & formatting ----
\setlength{\parindent}{0pt}

% ---- Global Configurations ----
% Year for which you want automatic mapping:
\providecommand{\DiaryYear}{2029} % <-- set the year here

% Mode: "iso" (default) or "jan1"
% - "iso": ISO week numbering (week 1 is the ISO week)
%     mapping rule: week assigned to the month containing the week's Monday.
% - "jan1": week 1 is the week that contains Jan 1, weeks start on Monday; mapping by Monday's month.
\providecommand{\WeekMode}{iso}

% ---- Helper Commands ----

% Month names
\providecommand{\MonthName}[1]{% 1..12
  \ifcase#1\relax\or January\or February\or March\or April\or May\or June\or July\or August\or September\or October\or November\or December\fi}

% Week date calculation helpers
\newcount\julianfirstweek
\newcount\wdayfirstweek
\newcount\mondayfirstweek
\newcount\julianmonday
\newcount\juliansunday

% Calculate week start date (Monday) given ISO week number
% Sets \WeekStartDay, \WeekStartMonth, \WeekStartYear
\providecommand{\CalcWeekStartDate}[2]{%
  \pgfcalendardatetojulian{#1-01-04}{\julianfirstweek}%
  \pgfcalendarjuliantoweekday{\julianfirstweek}{\wdayfirstweek}%
  \mondayfirstweek=\julianfirstweek
  \advance\mondayfirstweek by -\wdayfirstweek%
  \pgfmathtruncatemacro{\weekoffset}{(#2-1)*7}%
  \julianmonday=\mondayfirstweek
  \advance\julianmonday by \weekoffset\relax%
  \pgfcalendarjuliantodate{\julianmonday}{\tmpyear}{\tmpmonth}{\tmpday}%
  \xdef\WeekStartDay{\tmpday}%
  \xdef\WeekStartMonth{\tmpmonth}%
  \xdef\WeekStartYear{\tmpyear}%
}

% Calculate week end date (Sunday) given ISO week number
% Sets \WeekEndDay, \WeekEndMonth, \WeekEndYear
\providecommand{\CalcWeekEndDate}[2]{%
  \pgfcalendardatetojulian{#1-01-04}{\julianfirstweek}%
  \pgfcalendarjuliantoweekday{\julianfirstweek}{\wdayfirstweek}%
  \mondayfirstweek=\julianfirstweek
  \advance\mondayfirstweek by -\wdayfirstweek%
  \pgfmathtruncatemacro{\weekoffset}{(#2-1)*7 + 6}%
  \juliansunday=\mondayfirstweek
  \advance\juliansunday by \weekoffset\relax%
  \pgfcalendarjuliantodate{\juliansunday}{\tmpyear}{\tmpmonth}{\tmpday}%
  \xdef\WeekEndDay{\tmpday}%
  \xdef\WeekEndMonth{\tmpmonth}%
  \xdef\WeekEndYear{\tmpyear}%
}

% Format week date range
% Sets \WeekDateRange
\providecommand{\FormatWeekDates}[2]{%
  \CalcWeekStartDate{#1}{#2}%
  \CalcWeekEndDate{#1}{#2}%
  \xdef\WeekDateRange{\WeekStartDay.\WeekStartMonth. - \WeekEndDay.\WeekEndMonth.}%
}


% Build week-to-month mapping
\foreach \w in {1,...,53}{%
  \CalcWeekStartDate{\DiaryYear}{\w}%
  \xdef\WeekMonthNum{\the\numexpr\WeekStartMonth\relax}%
  \ifcsdef{WeeksForMonth\WeekMonthNum}%
    {\csxdef{WeeksForMonth\WeekMonthNum}{\csuse{WeeksForMonth\WeekMonthNum},\w}}%
    {\csxdef{WeeksForMonth\WeekMonthNum}{\w}}%
}


% Build week-to-date mapping for quick lookup
\foreach \w in {1,...,53}{%
  \CalcWeekStartDate{\DiaryYear}{\w}%
  \csxdef{WeekStartDay\w}{\WeekStartDay}%
  \csxdef{WeekStartMonth\w}{\WeekStartMonth}%
}


\newcommand{\ClickBox}[2]{% #1=label target, #2=visible text
  \hyperref[#1]{\kern2pt\raisebox{0pt}[13pt][9pt]{\mbox{#2}}\kern2pt}%
}
  \begin{document}
    \newcommand{\monthboxwidth}{0.31\textwidth}
    \newcommand{\monthboxheight}{7.3cm}

\fi


\pagecolor{brown!15!white}

% Defaults (can be overridden before \input by main.tex)
\makeatletter
\@ifundefined{c@rowcount}
  {\newcounter{rowcount}}
  {}
\makeatother
\setcounter{rowcount}{0}
\newcount\FoundWeekNum

\newcount\julianday
\providecommand{\MonthNumber}{1}
\providecommand{\MonthTitle}{January}
\providecommand{\DiaryYear}{2026}
\providecommand{\WeekList}{1,2,3,4}
\providecommand{\MonthName}{January1}

\providecommand{\ThisWeekNumber}{1}

\newcount\julianday
\newcount\WeekStartMonth
\newcount\WeekStartDay
\providecommand{\CalcWeekStartDate}[2]{%
  \WeekStartMonth=0\relax
  \WeekStartDay=0\relax
}

\newcommand{\calendarboxheight}{0.6cm}
\newcommand{\calendarboxwidth}{0.87cm}
\newcount\FoundWeekNum

% A simple calendar for the month
\newcommand*{\calendarmonth}[2]{%
  \setcounter{rowcount}{1}%
  \begin{tikzpicture}[x=1.12cm,y=1.34cm]
    % display the month name at the top
    \path (3,0.4) node {\large\textbf{\hyperref[month-#2]{\MonthName{#2}}}};
    \begin{scope}
      [every node/.style={rectangle,fill=green!5,minimum width=1.12cm}]
      \foreach \x in {0,...,6}
       {\path (\x,-0.2) node {\pgfcalendarweekdayshortname{\x}};}
    \end{scope}
    \pgfcalendar{cal}{#1-#2-01}{#1-#2-last}
    {%
      % Is this the first day of the month?
      \ifnum\pgfcalendarcurrentday=1\relax
      % Fill in days from previous month if this isn't a Monday
      \ifdate{Monday}{}
      {
        % Get last day of previous month
        \julianday = \pgfcalendarcurrentjulian\relax
        \advance\julianday by -\pgfcalendarcurrentweekday\relax
        \foreach \x in {0,...,\numexpr\pgfcalendarcurrentweekday-1}
        {
          \pgfcalendarjuliantodate
            {\julianday}{\theyear}{\themonth}{\theday}
          \path (\x,-1)
           node
           [
             rectangle split,
             rectangle split parts=2,
             draw]
          {\number\theday
           \nodepart{two}
           \parbox[t][\calendarboxheight]{\calendarboxwidth}{\mbox{}}%
          };
          \global\advance\julianday by 1\relax
        }
      }
      \fi
      \def\thebackground{magenta!4}%
      \ifcsdef{\pgfcalendarsuggestedname}%
      {%
        \def\thecontents{\csuse{\pgfcalendarsuggestedname}}%
        \def\thebackground{black!10}%
      }%
      {%
        \def\thecontents{\mbox{}}%
        \ifdate{weekend}{\def\thebackground{black!10}}{}%
      }%
      \path (\pgfcalendarcurrentweekday,-\therowcount)
       node
       [
         rectangle split,
         rectangle split parts=2,
         rectangle split part fill={cyan!20,\thebackground},
         draw] 
      {\pgfcalendarcurrentday
       \nodepart{two}%
       \parbox[t][\calendarboxheight]{\calendarboxwidth}{\small\thecontents}%
      };
        \ifdate{Monday}{
        {\FoundWeekNum=0\relax
        \foreach \w in {1,...,53}{%
            \ifnum\csuse{WeekStartMonth\w}=#2\relax
            \ifnum\csuse{WeekStartDay\w}=\pgfcalendarcurrentday\relax
                \global\FoundWeekNum=\w\relax
            \fi
            \fi
        }%
        \path (-1.1,-\therowcount) node  {\ClickBox{week-\the\FoundWeekNum}{Wk~\the\FoundWeekNum}};}      %%{\hyperref[week-\the\FoundWeekNum]{Wk~\the\FoundWeekNum}};}
        }{}%
      \ifdate{Sunday}{\stepcounter{rowcount}}{}%
      \xdef\lastjulianday{\number\pgfcalendarcurrentjulian}
      \xdef\lastweekday{\number\pgfcalendarcurrentweekday}
    }%
    \ifnum\lastweekday < 6\relax
     \julianday = \lastjulianday\relax
     \edef\lastweekday{\number\numexpr\lastweekday+1\relax}
     \foreach \x in {\lastweekday,...,6}
     {
       \global\advance\julianday by 1\relax
       \pgfcalendarjuliantodate{\julianday}{\theyear}{\themonth}{\theday}
       \path (\x,-\therowcount)
        node
        [
          rectangle split,
          rectangle split parts=2,
          draw]
       {\number\theday
        \nodepart{two}
        \parbox[t][\calendarboxheight]{\calendarboxwidth}{\mbox{}}%
       };
     }
    \fi
  \end{tikzpicture}%
  }
\renewcommand{\monthboxwidth}{0.31\textwidth}
\renewcommand{\monthboxheight}{5.3cm}

\clearpage

% Define length for storing calendar height
\newlength{\measuredcalheight}
\newlength{\totalcolumnheight}
\newlength{\fboxheight}

\setlength{\totalcolumnheight}{0.86\textheight} % Adjust as needed

% Command to create a column with calendar + fbox
\newcommand{\quartercolumn}[1]{%
  \begin{minipage}[c]{\monthboxwidth}
    % Measure the calendar height
    \setbox0=\vbox{\calendarmonth{\DiaryYear}{#1}}%
    \measuredcalheight=\ht0%
    \advance\measuredcalheight by \dp0%
    % Display the calendar
    \usebox{0}\\[0pt]%
    % Calculate remaining height
    \fboxheight=\totalcolumnheight%
    \advance\fboxheight by -\measuredcalheight%
    \advance\fboxheight by -6pt% % Subtract spacing
    % Create fbox with remaining height
    \begin{minipage}[t][\fboxheight][t]{0.97\textwidth} \color{black!20}\foreach \i in {1,...,12}{\rule{1.1\textwidth}{0.4pt}\\[12pt]}\end{minipage}%
  \end{minipage}%
}
{\small
\phantomsection
\label{q1-overview}
{\Large \ Quarter 1  \ $-$  \ClickBox{annual-overview}{\phantom{~}\DiaryYear\phantom{~}}}
\\[-8pt]
\noindent 
\quartercolumn{1}\hfill
\quartercolumn{2}\hfill
\quartercolumn{3} \ \ \hfill

\clearpage
\phantomsection
\label{q2-overview}
{\Large \ Quarter 2 \ $-$  \ClickBox{annual-overview}{\phantom{~}\DiaryYear\phantom{~}}}
\\[-8pt]
\noindent 
\quartercolumn{4}\hfill
\quartercolumn{5}\hfill
\quartercolumn{6} \ \ \hfill

\clearpage
\phantomsection
\label{q3-overview}
{\Large \ Quarter 3 \ $-$ \ClickBox{annual-overview}{\phantom{~}\DiaryYear\phantom{~}}}
\\[-8pt]
\noindent 
\quartercolumn{7}\hfill
\quartercolumn{8}\hfill
\quartercolumn{9} \ \ \hfill

\clearpage
\phantomsection
\label{q4-overview}
{\Large \ Quarter 4 \ $-$  \ClickBox{annual-overview}{\phantom{~}\DiaryYear\phantom{~}}}
\\[-8pt]
\noindent 
\quartercolumn{10}\hfill
\quartercolumn{11}\hfill
\quartercolumn{12} \ \ \hfill
}
\normalsize

\ifdefined\INCLUDEMODE\else
  \end{document}
\fi
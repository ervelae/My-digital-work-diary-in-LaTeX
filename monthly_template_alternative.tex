% ---- File: Monthly_template.tex ----
% A single-instance monthly page with links to weeks. Compiles standalone.
\ifdefined\INCLUDEMODE\else
  \documentclass[11pt]{article}
  \usepackage[paperwidth=288mm,paperheight=162mm,margin=2mm]{geometry}
  \usepackage{hyperref}
  \usepackage{pgffor}
  \usepackage{parskip}
  \usepackage{pgfcalendar}
  \usepackage{xcolor}
  \usepackage{tikz}
  \usepackage{tabularx}
  \usepackage{etoolbox}
  \usepackage{colortbl}
  \usetikzlibrary{calendar,shapes.multipart}
  \setlength{\parindent}{0pt}
  \hypersetup{colorlinks=true,linkcolor=blue}
  
  % ---- configuration ----
  % Year for which you want automatic mapping:
  \newcommand{\DiaryYear}{2026} % <-- set the year here
  % Mode: "iso" (default) or "jan1"
  % - "iso": ISO week numbering (week 1 is the ISO week)
  %     mapping rule: week assigned to the month containing the week's Monday.
  % - "jan1": week 1 is the week that contains Jan 1, weeks start on Monday; mapping by Monday's month.
  \newcommand{\WeekMode}{iso} % or jan1
  \newcommand{\dayentries}{}

  \begin{document}

  
\fi

\pagecolor{red!5!white}


% Defaults (can be overridden before \input by main.tex)
\makeatletter
\@ifundefined{c@rowcount}
  {\newcounter{rowcount}}
  {}
\makeatother
\setcounter{rowcount}{0}

\newcount\julianday
\providecommand{\MonthNumber}{1}
\providecommand{\MonthTitle}{}
\providecommand{\DiaryYear}{2026}
\providecommand{\MonthName}[1]{% 1..12
  \ifcase#1\relax\or January\or February\or March\or April\or May\or June\or July\or August\or September\or October\or November\or December\fi}
\providecommand{\WeekList}{1,2,3,4}

\providecommand{\ThisWeekNumber}{} % avoid "Undefined control sequence"

\newcount\julianday
\newcount\WeekStartMonth
\newcount\WeekStartDay
\providecommand{\CalcWeekStartDate}[2]{%
  % fallback: set numeric registers to 0 so \ifnum tests don't fail
  \WeekStartMonth=0\relax
  \WeekStartDay=0\relax
}



%\AtBeginDocument{%
%  \phantomsection
%  \label{month-\MonthNumber}%
%}


\phantomsection
\label{month-\MonthNumber}

% If MonthTitle not provided, use MonthName
\ifx\MonthTitle\empty
  \edef\MonthTitleText{\MonthName{\MonthNumber}}
\else
  \edef\MonthTitleText{\MonthTitle}
\fi









% A simple calendar for the month

\newcommand*{\calendarmonth}[2]{%
  \fbox{%
  \setcounter{rowcount}{1}%
  \begin{tikzpicture}[x=1.5cm,y=1.75cm]
    % display the month name at the top
    \path (3,1) node {\pgfcalendarmonthname{#2}};
    \begin{scope}
      [every node/.style={rectangle,fill=green!5,minimum width=1.4cm}]
      \foreach \x in {0,...,6}
       {\path (\x,0) node {\pgfcalendarweekdayshortname{\x}};}
    \end{scope}
    \pgfcalendar{cal}{#1-#2-01}{#1-#2-last}
    {%
      % Is this the first day of the month?
      \ifnum\pgfcalendarcurrentday=1\relax
      % Fill in days from previous month if this isn't a Monday
      \ifdate{Monday}{}
      {
        % Get last day of previous month
        \julianday = \pgfcalendarcurrentjulian\relax
        \advance\julianday by -\pgfcalendarcurrentweekday\relax
        \foreach \x in {0,...,\numexpr\pgfcalendarcurrentweekday-1}
        {
          \pgfcalendarjuliantodate
            {\julianday}{\theyear}{\themonth}{\theday}
          \path (\x,-1)
           node
           [
             rectangle split,
             rectangle split parts=2,
             draw]
          {\number\theday
           \nodepart{two}
           \parbox[t][1cm]{1.2cm}{\mbox{}}%
          };
          \global\advance\julianday by 1\relax
        }
      }
      \fi
      \def\thebackground{magenta!4}%
      \ifcsdef{\pgfcalendarsuggestedname}%
      {%
        \def\thecontents{\csuse{\pgfcalendarsuggestedname}}%
        \def\thebackground{black!4}%
      }%
      {%
        \def\thecontents{\mbox{}}%
        \ifdate{weekend}{\def\thebackground{black!4}}{}%
      }%
      \path (\pgfcalendarcurrentweekday,-\therowcount)
       node
       [
         rectangle split,
         rectangle split parts=2,
         rectangle split part fill={cyan!20,\thebackground},
         draw] 
      {\pgfcalendarcurrentday
       \nodepart{two}%
       \parbox[t][1cm]{1.2cm}{\small\thecontents}%
      };
      \ifdate{Sunday}{\stepcounter{rowcount}}{}%
      \xdef\lastjulianday{\number\pgfcalendarcurrentjulian}
      \xdef\lastweekday{\number\pgfcalendarcurrentweekday}
    }%
    \ifnum\lastweekday < 6\relax
     \julianday = \lastjulianday\relax
     \edef\lastweekday{\number\numexpr\lastweekday+1\relax}
     \foreach \x in {\lastweekday,...,6}
     {
       \global\advance\julianday by 1\relax
       \pgfcalendarjuliantodate{\julianday}{\theyear}{\themonth}{\theday}
       \path (\x,-\therowcount)
        node
        [
          rectangle split,
          rectangle split parts=2,
          draw]
       {\number\theday
        \nodepart{two}
        \parbox[t][1cm]{1.2cm}{\mbox{}}%
       };
     }
    \fi
  \end{tikzpicture}%
  }
}

\calendarmonth{\DiaryYear}{\MonthNumber} 


\clearpage




\ifdefined\INCLUDEMODE\else
  \end{document}
\fi
